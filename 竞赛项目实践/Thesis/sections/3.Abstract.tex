\begin{summary}
    近年来,随着高校教师职称评定对课堂教学评价的重视,如何确保教学评分的科学性与公平性成为亟需解决的问题。
    本文针对某校教师教学评价体系,从专家组集中评审到学院分散评比的改革入手,综合利用数学统计方法和标准化处理手段,分析评审数据的分布规律与偏差来源。
    首先,采用配对样本t检验等统计方法,对2023年由学校统一组织两组专家对同一批教师的评价结果进行差异性分析,判断两组评分的显著性差异及结果可信性。
    其次,针对2024年各学院分别评分导致的极差差异和标准不统一等问题,结合描述性统计分析、标准化转换(如Z-score标准化、极差调整等)探究各学院评分分布特征,提出基于归一化方法的全校教师评分汇总模型。
    通过该模型,有效消除学院之间评分尺度差异,实现了教师评分的公平可比。
    最终,通过对模型结果的合理性分析,证明了所提汇总方法的科学性与实用性,为高校教师评价体系的规范化与优化提供了数据支撑与理论参考。
    \vspace{0.5cm}

    \textbf{关键词:} 教师教学评价,评分标准化,统计分析,分组差异,评分汇总方法
\end{summary}
    