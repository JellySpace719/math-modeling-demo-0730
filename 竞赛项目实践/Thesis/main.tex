\documentclass[12pt]{article}
\usepackage{geometry}
% \usepackage{stmaryrd}
% \usepackage{imakeidx}
\geometry{left=1in,right=0.75in,top=1in,bottom=1in}

%%%%%%%%%%%%%%%%%%%%%%%%%%%%%%%%%%%%%%%%
% Replace ABCDEF in the next line with your chosen problem
% and replace 1111111 with your Team Control Number
\newcommand{\group}{本科}
\newcommand{\Number}{CM****}
%%%%%%%%%%%%%%%%%%%%%%%%%%%%%%%%%%%%%%%%
\usepackage[backend=bibtex]{biblatex}

% \usepackage{newtxtext}
\usepackage{amsmath,amssymb,amsthm}
% \usepackage{newtxmath} % must come after amsXXX
% \usepackage{tocloft}

% \usepackage[pdftex]{graphicx}
\usepackage{xcolor}
\usepackage{fancyhdr}
%-----数学宏包-----
% \usepackage{mathrsfs,bm}
%-----draft下 label提示-----
% \usepackage[notcite,notref]{showkeys}
% %-----设置超链接-----
\usepackage{url,hyperref}
\hypersetup{colorlinks=true,linkcolor=black,citecolor=black} % 去掉目录红框
%-----制作目录-----
% \usepackage{imakeidx}
% 设置颜色
% \usepackage{color,xcolor}
% 插入图片
% \usepackage{graphicx}
% \usepackage{epsfig}
%-----设置表格-----
% \usepackage{tabularx,array}
% \usepackage{longtable}
% \usepackage{booktabs}
% \usepackage{multirow}
% \usepackage{multicol}
%-----调整单元格格式-----
% \usepackage{makecell}
%-----操作字符串-----
\usepackage{xstring}
%-----多语种处理-----
% \usepackage[english]{babel}
%-----设置代码环境-----
% \usepackage{listings}
%-----设置章节标题和目录-----
\usepackage{titling}
\usepackage{titletoc}
\usepackage{titlesec}
%-----数学公式扩展-----
\usepackage{mathtools}
%-----中文支持-----
\usepackage{fontspec}
\usepackage{xeCJK}
\setCJKmainfont{SimSun}           % Win默认宋体
\setCJKsansfont{SimHei}           % 黑体
\setCJKmonofont{FangSong}         % 仿宋
\setmainfont{Times New Roman}     % 英文
\xeCJKsetup{AutoFakeBold=true, AutoFakeSlant=true}
%-----浮动体设置-----
\usepackage{float}
%-----书签设置-----
\usepackage{bookmark}
%-----数学宏包-----
\usepackage{amsmath,amsthm,amssymb,amsfonts}
\usepackage{mathrsfs,bm}
%-----draft下 label提示-----
% \usepackage[notcite,notref]{showkeys}
% --- 算法宏包及设置 ---
\usepackage{algorithm}
\usepackage{algpseudocode}
% ---- 定义列表项的样式 -----
\usepackage{enumitem}
\setlist{nolistsep}
% --- 设置英文字体 -----
\usepackage{newtxtext}  % for text fonts
% --- 设置数学字体 -----
\usepackage{newtxmath}
% --- 直接插入 pdf 文件 ----
\usepackage{pdfpages}
\usepackage{lastpage} % 用于获取总页数
\usepackage[titletoc,title]{appendix} % 用于添加附录
\usepackage{setspace} % 用于设置行间距
% 插入图片
\usepackage{graphicx}
\usepackage{epsfig}
%-----设置表格-----
\usepackage{tabularx,array}
\usepackage{longtable}
\usepackage{booktabs}
\usepackage{multirow}
\usepackage{multicol}
%-----调整单元格格式-----
\usepackage{makecell}
%-----操作字符串-----
\usepackage{xstring}
%-----多语种处理-----
%\usepackage[english]{babel}
%-----设置代码环境-----
\usepackage{listings}
%-----设置章节标题和目录-----
\usepackage{titletoc}
\usepackage{titlesec}
%-----数学公式扩展-----
\usepackage{mathtools}
%-----浮动体设置-----
\usepackage{float}
%-----书签设置-----
\usepackage{bookmark}
\usepackage{algorithm}
\usepackage{algpseudocode}
\usepackage{setspace}
\usepackage{minted} % 用于插入代码
\renewcommand*{\baselinestretch}{1.5}
\newcommand{\upcite}[1]{\textsuperscript{\textsuperscript{\cite{#1}}}}
\definecolor{bg}{rgb}{0.95, 0.95, 0.95}
% 设置 minted 环境的全局选项
\setminted{
    linenos, % 显示行号
    frame=lines, % 添加边框
    framesep=2mm, % 设置边框与代码之间的间距
    breaklines, % 自动换行
    bgcolor=bg, % 设置背景颜色
    baselinestretch = 1.0,
    numbersep = 5pt,
    numberblanklines = false,
	tabsize = 4
}
% ----- 设置浮动体间距 ------
\setlength{\textfloatsep}{0pt}
\setlength{\floatsep}{10pt plus 3pt minus 2pt}
\setlength{\intextsep}{10pt}
\setlength{\abovecaptionskip}{2pt plus1pt minus1pt}
\setlength{\belowcaptionskip}{3pt plus1pt minus2pt}

% ----- 设置公式间距为零 ------
\AtBeginDocument{
	\setlength{\abovedisplayskip}{4pt plus1pt minus1pt}
	\setlength{\belowdisplayskip}{4pt plus1pt minus1pt}
	\setlength{\abovedisplayshortskip}{2pt}
	\setlength{\belowdisplayshortskip}{2pt}
	\setlength{\arraycolsep}{2pt}   % array中列之间空白长度
}
% ---- 定义列表项的样式 -----
\usepackage{enumitem}
\setlist{nolistsep}
% \setlength{\itemsep}{3pt plus1pt minus2pt}

% --- 设置英文字体 -----
% \usepackage{newtxtext}  % for text fonts
\usepackage{fontspec}


% --- 自定义命令 -----
\newcommand{\CC}{\ensuremath{\mathbb{C}}}
\newcommand{\RR}{\ensuremath{\mathbb{R}}}
\newcommand{\A}{\mathcal{A}}
\newcommand{\ii}{\bm{\mathrm{i}}\,}  % 虚部
\newcommand{\md}{\mathrm{d}\,}
\newcommand{\bA}{\boldsymbol{A}}
\newcommand{\red}[1]{\textcolor{red}{#1}}


\lhead{Team \Team}
\rhead{}
\cfoot{}

\newtheorem{theorem}{Theorem}
\newtheorem{corollary}[theorem]{Corollary}
\newtheorem{lemma}[theorem]{Lemma}
\newtheorem{definition}{Definition}

% 定义 summary 环境
\newenvironment{summary}
  {\begin{center}\bfseries 摘要\end{center}\normalfont}

% 重新定义目录标题并居中显示
\renewcommand{\contentsname}{\centering 目录}
% 设置页眉
\pagestyle{fancy}
\fancyhf{}
\fancyhead[L]{参赛编号 \# 123456}
\fancyhead[R]{ \thepage\ / \pageref{LastPage}}
%%%%%%%%%%%%%%%%%%%%%%%%%%%%%%%%




\begin{document}

\onehalfspacing

\thispagestyle{empty}

\vspace*{-16ex}
\centerline{\begin{tabular}{*3{c}}
	\parbox[t]{0.3\linewidth}{\begin{center}\textbf{Problem Chosen}\\ \Large \textcolor{red}{\Problem}\end{center}}
	& \parbox[t]{0.3\linewidth}{\begin{center}\textbf{2025\\ MCM / ICM\\ Summary Sheet}\end{center}}
	& \parbox[t]{0.3\linewidth}{\begin{center}\textbf{Team Control Number}\\ \Large \textcolor{red}{\Team}\end{center}}	\\
	\bottomrule
\end{tabular}}

\begin{center}
    \Large{\textbf{B题:教师的教学评价模型建立与求解}}
\end{center}

\begin{summary}
    近年来,随着高校教师职称评定对课堂教学评价的重视,如何确保教学评分的科学性与公平性成为亟需解决的问题。
    本文针对某校教师教学评价体系,从专家组集中评审到学院分散评比的改革入手,综合利用数学统计方法和标准化处理手段,分析评审数据的分布规律与偏差来源。
    首先,采用配对样本t检验等统计方法,对2023年由学校统一组织两组专家对同一批教师的评价结果进行差异性分析,判断两组评分的显著性差异及结果可信性。
    其次,针对2024年各学院分别评分导致的极差差异和标准不统一等问题,结合描述性统计分析、标准化转换(如Z-score标准化、极差调整等)探究各学院评分分布特征,提出基于归一化方法的全校教师评分汇总模型。
    通过该模型,有效消除学院之间评分尺度差异,实现了教师评分的公平可比。
    最终,通过对模型结果的合理性分析,证明了所提汇总方法的科学性与实用性,为高校教师评价体系的规范化与优化提供了数据支撑与理论参考。
    \vspace{0.5cm}

    \textbf{关键词:} 教师教学评价,评分标准化,统计分析,分组差异,评分汇总方法
\end{summary}
    

\clearpage



%%%%%%%%%%%%%目录%%%%%%%%%%%%%%

\tableofcontents

%%%%%%%%%%%%%%%%%%%%%%%%%%%%%
\clearpage
\pagestyle{fancy}

%%%%%%%%%%%%%%%%%%%%%%%%%%%%%%

\section{Introduction}

\subsection{Background}

Juneau, the capital city of Alaska, seamlessly combines 
breathtaking natural beauty with a rich cultural heritage. 
Nestled in the southeastern part of the state, this unique city 
is accessible only by air or sea, giving it an island-like allure 
despite being located on the mainland. Home to approximately 30,000 
residents, Juneau welcomes over a million tourists annually—a number 
that continues to grow each year. While tourism has significantly 
boosted the city’s economy, it has also brought challenges, such 
as receding glaciers, increasingly crowded streets, and rising 
carbon emissions. To ensure its long-term prosperity, Juneau 
must embrace a \textbf{sustainable tourism strategy} that balances growth 
with the preservation of its natural and cultural treasures, which will be
presented in the following sections.


\subsection{Restatement and Analyses of the Problem}

We need to complete the following tasks based on the given background
and our collected data.

\begin{itemize}
    \item \textbf{Task 1: Develop a model to quantify the tourism industry in Juneau and analyse the model.}
    \begin{itemize}
        \item The model is required to qualitatively and quantitatively analyze the factors that affect the tourism industry in Juneau, including the economy, society, and environment. 
        \item The model should be able to predict the number of tourists in the next few years and provide insights into the development of the tourism industry in Juneau.
        \item A sensitivity anslysis should be conducted to evaluate the robustness of the model.
    \end{itemize}
    \item \textbf{Task 2: Test the model's adaptability and migration capability in Sitka, Alaska.}
    \\Based on the model developed in Task 1, we need to adapt the model to the city of Sitka, Alaska, and test its adaptability and migration capability.
    \item \textbf{Task 3: Propose a sustainable tourism strategy for Juneau.}
    \\Based on the model developed in Task 1, we need to propose a sustainable tourism strategy for Juneau that balances economic growth with environmental and social sustainability.
\end{itemize}

It can be noted that task 1 serves as the foundation for Task 2 and Task 3, 
while Task 2 provides a practical application of the model developed in Task 1. 
Task 3 aims to address the challenges and opportunities identified in Task 1 and Task 2, 
providing a comprehensive and sustainable solution for the tourism industry in Juneau.

Questions can be asked to further clarify the problem:
How to quantify the tourism industry in Juneau? Which factors should be 
considered in the model and what methods should be used? After developing 
the model, how can we adapt it to another city? What suggestions and 
strategies can be proposed to promote the sustainable development of the
tourism industry in Juneau?

In summary, we should effectively build a model that can quantify the tourism
industry in Juneau, adapt the model to another city, and propose a sustainable
tourism strategy for Juneau.


\subsection{Overview of Our Work}

On the basis of the above analyses we carried out out work and the 
working framework is shown below.

\begin{figure}[H]
    \centering
    \includegraphics[width=1\textwidth]{FrameWork.jpg} % 插入图片
    \caption{Our Work Overview Schematic Diagram}
\end{figure}



\section{Assumptions and Notations}

\subsection{Assumptions}

The following reasonable assumptions are made to reasonably simplify the model:

\begin{itemize}
  \item Government policies (such as taxes, subsidies, regulations, etc.) remain unchanged during the period of the model.
  \item No major event compromising or promoting the tourism industry will occur during the period of our model.
  \item Consumer behavior, consumer preferences, or market demand are assumed to remain unchanged.
  \item Consumption per visitor per day remains stable.
\end{itemize}


\subsection{Notations}

The primary notations used in this paper are listed in Table 1.

\begin{table}[!htbp]
  \begin{center}
  \caption{Notations}
  \begin{tabular}{cc}
    \toprule
    \multicolumn{1}{m{3cm}}{\centering Symbol}
    &\multicolumn{1}{m{12cm}}{\centering Definition}\\
    \midrule
    $\eta$         &   tax rate\\
    $\mathcal{F}$         &   indicator of sustainability\\
    $Q$    &   amount of fine\\
    $\mathcal{I}$   &   social impact\\
    $N$    &   number of tourists (thousands)\\
    $N_{Local}$    &   number of local residents (thousands)\\
    $N_{Max}$    &   maximum number of tourists regulated (thousands)\\
    $CO_{2\_Tourism}$    &   carbon emissions from tourism\\
    $CG_{Tourism}$    &   tourism carbon emissions per GDP\\
    $CG_{All}$    &   carbon emissions per GDP\\
    \bottomrule
  \end{tabular}
  \end{center}
\end{table}

\section{Preliminary Analyses}


We first analyse the potential factors that may affect the tourism industry in Juneau, thus enabling
 a smoother transition to the model building process.

 \subsection{Number of Tourists}

 We found no existing data on the number of tourists visiting Juneau each year, but we can infer it
 by other means.

 According to [1] and [2], among all the transportation methods, cruise ships are the most popular way to visit Juneau,
 accounting for over 90\% of the total number of tourists. As the number of cruise ship passengers is available online, 
 we can use it as a proxy to estimate the total number of tourists.

According to [3], the number of cruise ship passengers visiting Juneau is as follows:

 \begin{table}[H]
    \centering
    \renewcommand{\arraystretch}{1.3}
    \caption{Number of Cruise Ship Visitors to Juneau}
 \begin{tabular}{|c|c|c|c|c|c|c|c|c|c|c|}
    \hline \textit{Year} & 2014 & 2015 & 2016 & 2017 & 2018 & 2019 & 2020 & 2021 & 2022 & 2023 \\
    \hline \begin{tabular}{l} 
    \textit{Num(in thousands)} 
    \end{tabular} & 961 & 983 & 1015 & 1072 & 1151 & 1306 & 0 & 117 & 1167 & 1670 \\
    \hline
    \end{tabular}
\end{table}
It can be easily noted that numbers plummeted in 2020 and 2021 due to the COVID-19 pandemic.
In this section, we use the \textit{SARIMAX} model including the pandemiuc factor to 
predict the number of tourists in the next few years.

\subsubsection{SARIMAX Model}

The \textit{SARIMAX} model, which stands for \textit{Seasonal AutoRegressive Integrated 
Moving Average with eXogenous regressors}, is an extension of the \textit{ARIMA}
 \textit{(AutoRegressive Integrated Moving Average)} model that incorporates seasonal 
 effects and external variables. Since we need to consider the impact factors during the 
 pandemic, \textit{SARIMAX} is used instead of \textit{ARIMA}.

\subsubsection{Parameters Setting}

\begin{itemize}
    \item \textbf{Pandemic Impact Factor}: Given the severity of the COVID-19 pandemic, different factors are set.
    In 2020, 2021 when the pandemic was at its peak, factors are set to 1, in 2021 set to 0.3, and in other years set to 0.
    \item \textbf{Order (p, d, q)}: The order of the ARIMA part of the model is set to (2, 1, 1) after conducting the ACF and PACF analysis(see Figure 1).
    \item \textbf{Enforce Stationarity}: The enforce$\_$stationarity parameter is set to True to ensure the model is stationary.
    \item \textbf{Enforce Invertibility}: The enforce$\_$invertibility parameter is set to True to ensure the model is invertible.
\end{itemize}

\subsubsection{Model Results}

The \textit{SARIMAX} model is trained on the data from 2014 to 2023 
and used to predict the number of tourists in the next few years.
The prediction result is lited as follows. The residuals, ACF and PACF plots are 
also shown in Figure 1.

\begin{figure}[H]
    \centering
    \includegraphics[width=1\textwidth]{Fig_Tourists.jpg} % 插入图片
    \vspace{-0.4cm}
    \caption{Tourist Prediction}
\end{figure}

It can be seen that the model correctly handles the plummet during the pandemic and
captures the trend of the revival of tourism. The exact number of 
tourists in the next few years is shown below, which will be utilized in the following sections.

\begin{table}[H]
    \centering
    \caption{Number of Tourists Prediction}
    \renewcommand{\arraystretch}{1.3}
 \begin{tabular}{|c|c|c|c|c|c|}
    \hline \textit{Year} & 2024 & 2025 & 2026 & 2027 & 2028 \\
    \hline \begin{tabular}{l} 
    \textit{Num(in thousands)} 
    \end{tabular} & 1701 & 1774 & 1842 & 1909 & 1976 \\
    \hline
    \end{tabular}
\end{table}



\subsection{Number of Local Residents}

\subsubsection{Population of Juneau}

According to \textit{World Population Review}, the population of Juneau in the last decade is as follows:

\begin{table}[H]
    \centering
    \caption{Population of Juneau}
    \renewcommand{\arraystretch}{1.3}
    \resizebox{\textwidth}{!}{
    \begin{tabular}{|c|c|c|c|c|c|c|c|c|c|c|c|c|c|c|c|}
        \hline \textit{Year} & 2010 & 2011 & 2012 & 2013 & 2014 & 2015 & 2016 & 2017 & 2018 & 2019 & 2020 & 2021 & 2022 & 2023 & 2024 \\
        \hline \textit{Num (in thousands)} & 31.4 & 32.2 & 32.4 & 32.6 & 32.5 & 32.6 & 32.5 & 32.1 & 32.0 & 32.0 & 32.2 & 32.0 & 31.7 & 31.6 & 31.3 \\
        \hline
    \end{tabular}
    }
\end{table}

\subsubsection{Population Prediction}

We still use the \textit{SARIMAX} model proposed in the last section
to predict the population of Juneau in the next few years. Parameters are the same as the last section.
The first four pictures are still the original data and predicted data, the residual, ACF and PACF plots.
In addition, official prediction data can also be found in \textit{World Population Review}, therefore two additional pictures are added to compare the prediction results.

\begin{figure}[H]
    \centering
    \includegraphics[width=1\textwidth]{Fig_Population.jpg} % 插入图片
    \vspace{-0.4cm}
    \caption{Tourist Prediction}
\end{figure}

It can be concluded from subfigure 5 and 6 that the model fits the data well 
and the prediction is reliable. The exact number of local residents in the next 
few years is shown below.

\begin{table}[H]
    \centering
    \caption{Population Prediction}
    \renewcommand{\arraystretch}{1.3}
 \begin{tabular}{|c|c|c|c|c|c|}
    \hline \textit{Year}  & 2025 & 2026 & 2027 & 2028 & 2029\\
    \hline \begin{tabular}{l} 
    \textit{Num(in thousands)} 
    \end{tabular} & 31.2 & 30.8 & 30.7 & 30.4 & 30.0 \\
    \hline
    \end{tabular}
\end{table}


\clearpage


\section{Task 1: Model for Tourism Industry in Juneau}

\subsection{Introduction}


In this section we need to select factors to quantify and track the tourism industry in Juneau. 
It is impossible and unnecessary to consider all the factors that may affect the 
tourism industry, only those that are relevant to the problem need to be considered.
Drawing on the idea of the divide-and-conquer algorithm, we first divide the factors 
into three categories: economy, environment and society (aka hidden cost factors). 
The untimate goal is listed below:


\begin{equation}
    \mathcal{F}=(\alpha \cdot \text { Economy }-\beta \cdot \text { Environment }) / \text { Hidden Cost Factors }
\end{equation}

where $\alpha$ and $\beta$ are the weights of the economy and environment, denoting the importance we attach to each category.
Economy means the income generated by the tourism industry, environment means the environmental cost of the tourism industry, and 
society is an indicator that quantifies the satisfaction of the local residents towards the tourism industry.

Our goal is to maximize the final output $\mathcal{F}$. Intuitively, it 
is equivalent to maximizing economy income, minimizing the environmental 
impact and elevating the final scores of hidden cost factors.

Each category is further divided into several minor factors such as local population, 
number of tourists to extrapolate a mathematical model fitting the circumstances in Juneau,
which will be discussed in the following sections.





\subsection{Economy}

In this section we consider the actions that will contribute to the income of the tourism industry in Juneau, which are
tourists' consumption, tax income and fines.

\subsubsection{Tourists' Consumption}

We first calculate the average consumption of tourists in Juneau per day. Since there is no existing official data available, 
we can infer it by other means. According to [5], avearge tax income from tourists in Juneau is 27.7 million dollars in 2018
with a tax rate of 12\%. We can use this information to estimate the average consumption of tourists in Juneau per day according to 
the following equation.

\begin{equation}
    \text{Average Consumption} = \frac{\text{Tax Income}}{\text{Tax Rate} \times \text{Number of Tourists}}
\end{equation}

The number of tourists can be found in Table 2. The average consumption of 
tourists in Juneau per day is calculated as follows:

\begin{equation}
    \text{Average Consumption} = \frac{27.7 \times 10^6}{0.12 \times 1151 \times 10^3} \approx 200.55
\end{equation}

Therefore the function of tourists' consumption regarding the number of tourists is:


\begin{equation}
    \text{Tourists' Consumption} = 200.55 N
\end{equation}

Given that an average of 3 days are spent by each visitor to Juneau, 
the total consumption of tourists should multiply by another 3.

\subsubsection{Tax Income}

According to the official website of Juneau, the tax rate of the tourism industry is 12\%.
The tax income can be calculated as follows:

\begin{equation}
    \text{Tax Income} = 0.12 \times \text{Tourists' Consumption} \approx 24 N
\end{equation}

We should also consider the case when the tax rate is not fixed to propose suggestions to the
government on how to adjust the tax rate to maximize the income of the tourism industry. We
use the data calculated as above as the base case and assume that the tax rate $\eta$ is associated with the number of tourists.

The actual visitors $N$ is calculated as follows:


\begin{equation*}
    N= N_0 \cdot f(\eta)
\end{equation*}

Intuitively, $\eta$ should be negatively correlated with the number of tourists. When $eta$ is 
set to 0, we assume that the number of tourists will be $1.5N_0$, and when $\eta$ is set to 1, the number of tourists will be 0.
Specifically, when $\eta$ is set to the current tax rate 0.12, the number of tourists will be $N_0$. 
Also, $f(\eta)$ should be monotonically decreasing and fall steeply in the range of $[0,0.12]$ and $[0.85,1]$, 
Using the above data
we propose a model that fits the relationship between the tax rate and the number of tourists as follows:

\begin{equation*}
    f(\eta) = -5.5 \eta^3 + 9.1903 \eta^2 - 5.1903 \eta + 1.5, \quad 0 \leq \eta \leq 1
\end{equation*}

Therefore the tax income can be calculated as follows:

\begin{equation}
    \text{Tax Income} = 200.55N \cdot f(\eta), \quad 0 \leq \eta \leq 1
\end{equation}

\subsubsection{Fines}

As there is no official data available, we assume that the 
fine rate is negatively correlated with the amount of fines 
and follows an exponential distribution $f(x) = \lambda \cdot e^{-\lambda x}$. We also assume that 
fined rate falls to 5\% when the amount of fines climbs to 15 dollars, that is:

\begin{equation*}
    \int_0^{15} \lambda \cdot e^{-\lambda x} d x=1-95 \% \Rightarrow \lambda \approx 0.2
\end{equation*}

Therefore the total amount of fines can be calculated as follows:

\begin{equation}
    \text { Fines }=N Q \cdot\left(1-\int_0^Q 0.2 \cdot e^{-0.2 x} d x\right)=N Q \cdot e^{-0.2 Q}
\end{equation}




\subsection{Environment}

% According to the official website of Juneau, its tourism industry is 
% mainly comprised of glacier tours, whale watching, rainforest tours and others.
% We assume each of these activities accounts for a certain percentage of the total environmental impact,
% denoted as $v_1$, $v_2$, $v_3$ and $v_4$ respectively. Due to the receding of glaciers,
% our goal is to lower the percentage of glacier tours and increase the percentage of other activities.

In this section we propose a new model $KAYA_{tourism}$ derived 
from the KAYA model to quantify the environmental impact of the tourism industry in Juneau.

\subsubsection{KAYA Model}

The original KAYA model is a mathematical model that 
describes the relationship between the total 
amount of CO2 emissions and the four factors 
that affect it: population, GDP per capita, energy intensity and carbon intensity.
 The KAYA model is expressed as follows:

\begin{equation*}
    \text { CO2 Emissions }=P \times GDP \times E I \times C I
\end{equation*}

where $P$ denotes the population, $GDP$ denotes the GDP per capita, 
$EI$ denotes the energy intensity and $CI$ denotes the carbon intensity.
This model falls short when only considering the environmental impact of the tourism industry.
Based on the data we collect and the goal of our project, we propose a new model $KAYA_{tourism}$.


\subsubsection{$KAYA_{Tourism}$ Model}

The $KAYA_{Tourism}$ model is expressed as follows:

\begin{equation}
    CO_{2\_Tourism}=G \times CG_{Tourism}
\end{equation}

where $G$ denotes the gross income of the tourism industry and $CG = EI \times CI$. To calculate 
the emission of $CO_2$ in the tourism industry, we first looked up the data of the Juneau's carbon emission and
GDP across the country and calculated the $CG_{All}$ across the country. $CG_{Tourism}$ can 
be calculated as $CG_{Tourism} = CG_{All} \times Ratio $  where $Ratio$ is the 
ratio tourism accounts for across all industries. The income of the tourism industry in Juneau 
has been calculatd in the previous section, and the $CO_{2\_Tourism}$ can thus be estimated.
It should be noted that $G$ and $CG_{Tourism}$ are both linear functions of the number of tourists $N$,
therefore when fitting and regressing $CO_{2\_Tourism}$ quadratic regression should be used.

Alongside the historical data, future predictions are also conducted using the \textit{SARIMAX} model.
The results are listed below.

\begin{figure}[H]
    \centering
    \begin{minipage}{0.32\textwidth}
        \centering
        \includegraphics[width=\textwidth]{Ratio.jpg}
        % \caption{Caption 1}
    \end{minipage}
    \begin{minipage}{0.32\textwidth}
        \centering
        \includegraphics[width=\textwidth]{CG_pred.jpg}
        % \caption{Caption 2}
    \end{minipage}
    \begin{minipage}{0.32\textwidth}
        \centering
        \includegraphics[width=\textwidth]{CG_pred2.jpg}
        % \caption{Caption 3}
    \end{minipage}
\end{figure}

\begin{figure}[H]
    \centering
    \begin{minipage}{0.32\textwidth}
        \centering
        \includegraphics[width=\textwidth]{GDP.jpg}
        % \caption{Caption 4}
    \end{minipage}
    \begin{minipage}{0.32\textwidth}
        \centering
        \includegraphics[width=\textwidth]{Emission.jpg}
        % \caption{Caption 5}
    \end{minipage}
    \begin{minipage}{0.32\textwidth}
        \centering
        \includegraphics[width=\textwidth]{Price.jpg}
        % \caption{Caption 6}
    \end{minipage}
\end{figure}

\begin{figure}[H]
    \centering
    \begin{minipage}{0.32\textwidth}
        \centering
        \includegraphics[width=\textwidth]{Cost.jpg}
        % \caption{Caption 7}
    \end{minipage}
    \begin{minipage}{0.32\textwidth}
        \centering
        \includegraphics[width=\textwidth]{Carbon_pred1.jpg}
        % \caption{Caption 8}
    \end{minipage}
    \begin{minipage}{0.33\textwidth}
        \centering
        \includegraphics[width=\textwidth]{Carbon_pred2.jpg}
        % \caption{Caption 9}
    \end{minipage}
\end{figure}

From subplot 9 we can see that a quadratic regression 
fits the data well with an $R^2$ value of over 0.99.
The fitting function yields the following results:

\begin{equation}
    CO_{2\_Tourism} = 0.815 \times \frac{N^2}{10^5} - 14.95N+7924000
\end{equation}



\subsection{Hidden Cost Factors}

Societal factors such as infrastructure, price of housing products, and the mental
loss due to the overcrowding and rowdy tourists all account for the hidden costs of the tourism industry.

\subsubsection{Data Processing}

Firstly we collected comprehensive and accurate data from all sources regarding Juneau, including the following factors:

\textit{Satisfy\_Score},\textit{Crowding\_at\_Mendenhall\_Glacier}, \textit{Crowding\_on\_sidewalks\_downtown}, \\ \textit{Vehicle\_congestion\_downtown}, \textit{Flightseeing\_noise}, \textit{Air\_emissions\_from\_cruise\_ships},\\ \textit{Vehicle\_congestion\_outside\_of\_downtown}, 
\textit{Whale\_watching\_boat\_traffic\_and\_wakes}, \textit{Crowding\_on\_trails}, \textit{Street\_Services}, \textit{Wastewater}, \textit{Public\_Transit}, \textit{Parks\_and\_Recreation}, 
\textit{Docks}, \textit{Ports}

We then use the data $N$ and $N_{Local}$ calculated in \textit{Section 3} to fit a linear regression model. The results are as follows:

\begin{figure}[htbp]
    \centering
    \begin{minipage}[t]{0.49\textwidth}
        \centering
        \includegraphics[width=1\textwidth]{2dLinear_1.jpg} % 插入图片
	    %   \vspace{-0.8cm}
        % \caption{}
    \end{minipage}
    \hfill
	%   \hspace{0.1\textwidth} % 调整这里的值来改变两张图片之间的间距
    \begin{minipage}[t]{0.49\textwidth}
        \centering
        \includegraphics[width=1\textwidth]{2dLinear_2.jpg} % 插入图片
	    %   \vspace{-0.8cm}
        % \caption{}
\end{minipage}
\caption{Bivariate Linear Regression}
\end{figure}

Only a fraction of results are shown here. The full results can be found in \textit{Appendix A}.

\textbf{\textit{Model for stsfy: }}

$\text{stsfy} = -4.693528652706527\times 10^{-5} N - 0.0060844912311383 \times N_{local} + 407.6518514$

\textbf{\textit{Model for Crowding\_at\_Mendenhall\_Glacier: }}

$\text{Crowding\_at\_Mendenhall\_Glacier} = 1.1573084349139379\times 10^{-5} N - 0.0017873857238289111 \times N_{local} + 100.15433801$


\subsubsection{Entropy Weight Method}

The Entropy Weight Method (EWM) is a quantitative technique commonly used 
to determine the weight or importance of various factors in multi-criteria 
decision-making. It is based on the concept of information entropy from 
information theory, which measures the degree of disorder or uncertainty in a system.
We mainly utilized the following equations.

\paragraph{Convert to Probability Matrix}


\[P_{i j}=\frac{X_{i j}}{\sum_{i=1}^n X_{i j}}\]


where:
\begin{itemize}
    \item $P_{i j}$ is the probability value of the $i$-th row and $j$-th column.
    \item $X_{i j}$ is the value of the $i$-th row and $j$-th column in the original data matrix.
    \item $\sum_{i=1}^n X_{i j}$ is the sum of all elements in the $j$-th column.
\end{itemize}

\paragraph{Calculate Information Entropy}


\[H_j=-k \sum_{i=1}^n P_{i j} \ln \left(P_{i j}+\epsilon\right)\]

where:

\begin{itemize}
    \item $H_j$ is the entropy value of the $j$-th index.
    \item $k=\frac{1}{\ln (n)}$ is the normalization coefficient to ensure that the entropy value is within the range of $[0,1]$.
    \item $P_{i j}$ is the probability value.
    \item $\epsilon$ is a very small value (e.g. $10^{-12}$) to avoid numerical errors in the logarithm calculation.
\end{itemize}

\vspace{0.5cm}

The calculated results are shown below:

\begin{table}[ht]
    \centering
    \renewcommand{\arraystretch}{1.3}
    \caption{Factors and Their Weights}
    \begin{tabular}{|l|c|l|c|}
    \hline
    \textbf{Factor} & \textbf{Weight} & \textbf{Factor} & \textbf{Weight} \\ \hline
    stsfy & 0.029112 & Wastewater & 0.038384 \\ \hline
    Crowding at Mendenhall Glacier & 0.029089 & Public Transit & 0.012025 \\ \hline
    Crowding on sidewalks downtown & 0.002553 & Parks and Recreation & 0.180182 \\ \hline
    Vehicle congestion downtown & 0.058313 & Docks & 0.024321 \\ \hline
    Flightseeing noise & 0.025751 & Ports & 0.121009 \\ \hline
    Air emissions from cruise ships & 0.139424 & Grocery Items & 0.015780 \\ \hline
    Vehicle congestion outside of downtown & 0.129145 & JNU Housing & 0.034101 \\ \hline
    Whale watching boat traffic and wakes & 0.021685 & Utilities & 0.002685 \\ \hline
    Crowding on trails & 0.034678 & Transportation & 0.034639 \\ \hline
    Street Services & 0.067125 & & \\ \hline
    \end{tabular}
    \end{table}

It can be noted that \textit{Air emissions from cruise ships} and \textit{Vehicle congestion outside of downtown} are the two important factors that affect the hidden cost of the tourism industry in Juneau, 
which denotes that our model is able to capture the essence of the problem and the need of people. It also provides a comprehensive and accurate solution.
    

\subsubsection{Social Impact Model}

Multiplying the weight of each factor by the corresponding value and summing them up, we can get the social impact $\mathcal{I}$.
The result yields:

\begin{equation}
    \mathcal{I}=5.011\times 10^{-9} N -0.002 \times N_{Local}+69.93
\end{equation}


\subsection{Summary}



Summing up all the three categories, we can get the final output of the model:

\begin{equation}
    \begin{aligned}
    &\left\{\begin{array}{l}
    \mathcal{F}=(\alpha \cdot \text { Economy }-\beta \cdot \text { Environment }) / \text { Society } \\[10pt]
    \text { Economy }=200.55 N \cdot f(\eta) \cdot (\eta+1)+NQ\cdot e^{-0.2 Q} \\[10pt]
    \text { Environment }=0.815 \cdot \frac{N^2}{10^5}-14.95 N+7924000 \\[10pt]
    \text { Society }=5.011\times 10^{-9} N -0.002 \times N_{Local}+69.93 \\[10pt]
    f(\eta)=-5.5 \eta^3+9.1903 \eta^2-5.1903 \eta+1.5, \quad 0 \leq \eta \leq 1
    \end{array}\right.
    \end{aligned}
\end{equation}

where $N,Q,\eta$ are independent variables, $\alpha$, $\beta$ 
and are parameters that can be adjusted accordingly, 
$N_{Local}$ is the local population of Juneau assumed fixed.
Our goal is to find the optimal value of $N,Q,\eta$ that maximizes the output $\mathcal{F}$.

Since the number of tourists in the coming year is related to the number of tourists in the previous year, 
we use the $N$ in 2024 to predict the parameters in 2025, and the results are shown below.

When $\alpha$ is set to 1 and $\beta$ is set to 30, 
the optimal value of $N,Q,\eta$ is calculated as follows:

\begin{equation}
    \begin{aligned}
    &\left\{\begin{array}{l}
    N_{Max} = 1408 \\[10pt]
    N=1264 \\[10pt]
    Q=5 \\[10pt]
    \eta=0.221
    \end{array}\right.
    \end{aligned}
\end{equation}

The results suggest a feasible and effective strategy that the government can implement to promote sustainable development. 

The proposed measures include reducing the maximum tourist capacity from 1.728 million in 2024 to 1.408 million, increasing the tax rate to 0.221, and introducing a 5\$ environmental penalty. 

This strategy not only alleviates the pressure on Juneau's local ecosystem but also reduces the strain on essential infrastructure, such as water supply and public transportation. Additionally, it contributes to enhancing the overall well-being and satisfaction of local residents.

\subsection{Sensitivity Analysis}

To evaluate the robustness of our model, we conduct a sensitivity analysis.
Since there isn't many factors that can be adjusted, we don't need to use the Principal Component Analysis (PCA).

The results are shown below:

\begin{figure}[H]
    \centering
    \includegraphics[width=0.97\textwidth]{Sensitivity_Analysis.png} % 插入图片
    \vspace{-0.3cm}
    \caption{Sensitivity Analysis}
\end{figure}

It can be noted that when $\eta$ (tax rate) and $q$ are held constant, 
the output $\mathcal{F}$ does not change significantly with the change of $N_{Max}$.
However, when $\eta$ becomes the independent variable, the output $\mathcal{F}$
budges significantly with the change of $\eta$. The result is similar when $q$ (fine amount) becomes the independent variable.
Therefore, we can conclude that $N_{Max}$ has minimal impact on the result, 
$\eta$ shows moderate sensitivity with a non-linear effect, 
and $q$ is the most sensitive, displaying significant influence on 
the output within its range.

\clearpage


\section{Task 2: Model Adaptation and Migration}

In the previous section, we have established a model to quantify 
the tourism industry in Juneau. 
Based on this model we take a further look at the city of Sitka, Alaska and
test the model's adaptability and migration capability.

The following analyses, calculations and predictions are identical to those done
in the area of Juneau. The model is adapted to the city of Sitka, and the optimal tax rate, number of tourists, and fine rate are calculated.

\subsection{Procedure}

We first calculated the CG of Sitka, Alaska, and utilized the results
to predict the ratio of tourists to local residents in the next few years.
The CG prediction is shown in subplot 2. After that, we extrapolate the relationship
between CG and the number of tourists, and forcast GDP in the next few years.
After that, according to equation (8), we forcast the amount of carbon emission
from tourism in the next few years. The results are shown in subplot 5. In addition, 
we assume that the price of carbon increases by 5\% each year, carbon price prediction 
can thus be obtained. The results are shown in subplot 4. Finally, we use the equation
$\text{Carbon Cost} = \text{Carbon Emission} \times \text{Carbon Price}$ to calculate the
carbon cost in the next few years. The results are shown in subplot 7 and 8.

\subsection{Results}

The results of the prediction are shown in the following table.

\begin{figure}[H]
    \centering
    \begin{minipage}[t]{0.32\textwidth}
        \centering
        \includegraphics[width=1\textwidth]{Ratio_Sitka.png}
    \end{minipage}
    \hfill
    \begin{minipage}[t]{0.32\textwidth}
        \centering
        \includegraphics[width=1\textwidth]{CG_Sitka.png}
    \end{minipage}
    \hfill
    \begin{minipage}[t]{0.32\textwidth}
        \centering
        \includegraphics[width=1\textwidth]{CG_Pred_Sitka.png}
    \end{minipage}
\end{figure}

\begin{figure}[H]
    \centering
    \begin{minipage}[t]{0.32\textwidth}
        \centering
        \includegraphics[width=1\textwidth]{GDP_Sitka.png}
    \end{minipage}
    \hfill
    \begin{minipage}[t]{0.32\textwidth}
        \centering
        \includegraphics[width=1\textwidth]{C_Emission_Sitka.png}
    \end{minipage}
    \hfill
    \begin{minipage}[t]{0.32\textwidth}
        \centering
        \includegraphics[width=1\textwidth]{C_Price_Sitka.png}
    \end{minipage}
\end{figure}



\begin{figure}[H]
    \centering
    \begin{minipage}[t]{0.5\textwidth}
        \centering
        \includegraphics[width=1\textwidth]{icc_analysis.png}
    \end{minipage}
    \hfill
    \begin{minipage}[t]{1\textwidth}
        \centering
        \includegraphics[width=1\textwidth]{normality_test.png}
    \end{minipage}
\end{figure}

\subsection{Analyses}

Summing up all the categories and the parameters yields the final equation.

\begin{equation}
    \begin{aligned}
    &\left\{\begin{array}{l}
    \mathcal{F}=(\alpha \cdot \text { Economy }-\beta \cdot \text { Environment }) / \text { Society } \\[10pt]
    \text { Economy }=165 N \cdot f(\eta) \cdot (\eta+1)+NQ\cdot e^{-0.2 Q} \\[10pt]
    \text { Environment }=2.897 \cdot N^2+ 7.67\times 10^5 N- 3.293\times 10^{10} \\[10pt]
    \text { Society }=5.011\times 10^{-9} N + 4.2 \\[10pt]
    f(\eta)=-5.5 \eta^3+9.1903 \eta^2-5.1903 \eta+1.5, \quad 0 \leq \eta \leq 1
    \end{array}\right.
    \end{aligned}
\end{equation}

When $\alpha$ is set to 1 and $\beta$ is set to 30, 
the optimal value of $N,Q,\eta$ is calculated as follows:

\begin{equation}
    \begin{aligned}
    &\left\{\begin{array}{l}
    N_{Max} = 93.353 \\[10pt]
    N_{LastYear} = 95.937 \\[10pt]
    N=52.04 \\[10pt]
    Q=4.89 \\[10pt]
    \eta=0.286
    \end{array}\right.
    \end{aligned}
\end{equation}

The strategy is the same as we have proposed in Juneau, 
that is to increase the tax rate and the fine rate, and to 
decrease the number of tourists. Thus our model can be successfully
adapted to Sitka, Alaska, proving its adaptability and migration capability.

\subsection{Sensitivity Analysis}

To evaluate the robustness of the migrated model, we conduct a sensitivity analysis.

The results are shown below:

\begin{figure}[H]
    \centering
    \includegraphics[width=1\textwidth]{Sensitivity_Analysis_Sitka.jpg} % 插入图片
    \vspace{-0.5cm}
    \caption{Sensitivity Analysis}
\end{figure}

It can be shown that when $N_{Max}$ is the independent variable, the outcome is immune to the change of $N_{Max}$,
while in other circumstances are sensitive to the changing of the independent variables.
The result is similar to what we conducted in Juneau.

\clearpage



\section{Task 3: Memo}
\begin{center}
    \Large\textbf{Recommendations for Sustainable Tourism in Juneau, Alaska}
\end{center}

{

Dear Tourist Council,

It is our great honour to present to you our recommendations for 
sustainable tourism in Juneau, Alaska. We have conducted a thorough 
analysis of the current situation in Juneau and have identified several key 
areas that need to be addressed in order to ensure the long-term sustainability 
of the tourism industry in the region. 
Our recommendations are based on the principles of sustainable tourism, 
which aim to balance the economic, social, and environmental impacts of tourism 
in order to ensure that it can continue to benefit both the local community and 
the environment for generations to come. Our approaches, findings and suggestions are as follows.

Firstly we summarized a general equation aiming to balance the economic, social, 
and environmental impacts of tourism. Then we looked into these aspects
and devised a model accordingly for each. \textit{SARIMAX, Linear-Regression} models 
were used to ensure the accuracy and reliability of our findings and suggestions.

Here are some findings based on our predictions.

\begin{itemize}
    \item The emission of carbon dioxide from tourism is increasing at an alarming rate, and it is 
    highly correlated with the square number of tourists.
    \item Current tax rates may not be optimal for maximizing tourism income and environmental sustainability.
    \item The number of tourists is higher than the optimal number that can be accommodated by the local environment.
\end{itemize}

Based on these findings, we put forward the following recommendations and measures.

\begin{itemize}
    \item Impose a carbon tax of around 20\% on tourism to reduce carbon emissions and encourage sustainable practices.
    \item Increase the fine amount to 20\$ for tourists who violate environmental regulations to deter harmful behavior.
    \item Set the upper limit of tourists to 1.5 million per year to protect the local environment and culture. Reduce the number of tourists by 10\% each year to around 1.2 million by 2030.
\end{itemize}

I hope you find our recommendations useful and that they will help to guide
the development of sustainable tourism in Juneau. We believe that by working
together, we can create a more sustainable future for the tourism industry in
the region. Thank you for your attention and consideration. Feel free to contact us for any further information.

\begin{flushright}
    Sincerely,\\
    Team \# 2503720 Members
    \end{flushright}

}

\clearpage

\begin{appendices}
    \section{Bivariate Linear Regression Results}

The Bivariate Linear Regression results are as follows:

\vspace{0.5cm}

    \textbf{\textit{Model for stsfy: }}

    $\text{stsfy} = -4.693528652706527 \times 10^{-5} N - 0.0060844912311383 \times N_{local} + 407.6518514$
    
    \textbf{\textit{Model for Crowding at Mendenhall Glacier: }}
    
    $\text{Crowding\_at\_Mendenhall\_Glacier} = 1.1573084349139379 \times 10^{-5} N - 0.0017873857238289111 \times N_{local} + 100.15433801$
    
    \textbf{\textit{Model for Crowding on sidewalks downtown: }}
    
    $\text{Crowding\_on\_sidewalks\_downtown} = 6.429491305077434 \times 10^{-6} N + 0.0023403412645394937 \times N_{local} - 25.69203444$
    
    \textbf{\textit{Model for Vehicle congestion downtown: }}
    
    $\text{Vehicle\_congestion\_downtown} = 2.3146168698278767 \times 10^{-5} N + 0.01642522855234218 \times N_{local} - 496.69132398$
    
    \textbf{\textit{Model for Flightseeing noise: }}
    
    $\text{Flightseeing\_noise} = -9.001287827108412 \times 10^{-6} N - 0.015276477770355294 \times N_{local} + 540.76884821$
    
    \textbf{\textit{Model for Air emissions from cruise ships: }}
    
    $\text{Air\_emissions\_from\_cruise\_ships} = -1.543077913218585 \times 10^{-5} N - 0.017616819034894794 \times N_{local} + 618.46088265$
    
    \textbf{\textit{Model for Vehicle congestion outside of downtown: }}
    
    $\text{Vehicle\_congestion\_outside\_of\_downtown} = 1.9288473915232255 \times 10^{-6} N - 0.02029789762063815 \times N_{local} + 683.19238967$
    
    \textbf{\textit{Model for Whale watching boat traffic and wakes: }}
    
    $\text{Whale\_watching\_boat\_traffic\_and\_wakes} = 1.4144880871170358 \times 10^{-5} N + 0.0011487507819868843 \times N_{local} - 12.92247577$
    
    \textbf{\textit{Model for Crowding on trails: }}
    
    $\text{Crowding\_on\_trails} = 1.2858982610154833 \times 10^{-6} N - 0.013531931747092107 \times N_{local} + 465.46159311$
    
    \textbf{\textit{Model for Street Services: }}
    
    $\text{Street\_Services} = 0.7716946724830306 \times N - 1098.7869298693258 \times N_{local} + 39720019.96870749$
    
    \textbf{\textit{Model for Wastewater: }}
    
    $\text{Wastewater} = 2.0506436350699593 \times N - 2977.4037361145515 \times N_{local} + 1.06567591 \times 10^8$
    
    \textbf{\textit{Model for Public Transit: }}
    
    $\text{Public\_Transit} = -0.11276464701428993 \times N - 1619.7640126090985 \times N_{local} + 58718764.15347296$
    
    \textbf{\textit{Model for Parks and Recreation: }}
    
    $\text{Parks\_and\_Recreation} = 2.7473696565739054 \times N - 2597.301898470375 \times N_{local} + 92393330.57040936$
    
    \textbf{\textit{Model for Docks: }}
    
    $\text{Docks} = 2.0782557994938906 \times N + 424.7535749514112 \times N_{local} - 14090559.71680191$
    
    \textbf{\textit{Model for Ports: }}
    
    $\text{Ports} = 0.708958939693741 \times N - 944.9238521963889 \times N_{local} + 33938861.95916443$
    
    \textbf{\textit{Model for JNU Composite: }}
    
    $\text{JNU\_Composite} = 9.644236957615973 \times 10^{-7} N + 0.003851051189680889 \times N_{local} + 4.69619483$
    
    \textbf{\textit{Model for Grocery Items: }}
    
    $\text{Grocery\_Items} = -2.1860270437263293 \times 10^{-5} N + 0.026042839700565677 \times N_{local} - 663.74708291$
    
    \textbf{\textit{Model for JNU Housing: }}
    
    $\text{JNU\_Housing} = -2.0702962002349386 \times 10^{-5} N + 0.0018641011281828252 \times N_{local} + 107.36835089$
    
    \textbf{\textit{Model for Utilities: }}
    
    $\text{Utilities} = 1.3951996132018008 \times 10^{-5} N + 0.008178540544050684 \times N_{local} - 142.54171473$
    
    \textbf{\textit{Model for Transportation: }}
    
    $\text{Transportation} = 1.6909562132353666 \times 10^{-5} N - 0.01094490247426114 \times N_{local} + 442.51994943$
    
    \textbf{\textit{Model for Healthcare: }}
    
    $\text{Healthcare} = -3.2147456525386777 \times 10^{-6} N + 0.006829829367730268 \times N_{local} - 59.05398278$
    
    \textbf{\textit{Model for Misc. Goods and Services: }}
    
    $\text{Misc.\_Goods\_and\_Services} = 1.9931423045740026 \times 10^{-5} N - 0.018744942079927586 \times N_{local} + 681.15469324$
    

  \end{appendices}

\clearpage

\addcontentsline{toc}{section}{References}

\begin{thebibliography}{99}
  \bibitem{mcdowell2016} McDowell Group. (2016). \textit{ Juneau Visitor Profile and Economic Impact Study}. 
  \bibitem{southeast2023} Southeast Conference. (2023).  \textit{Juneau Tourism Survey 2023}. 
  \bibitem{juneau2016} City and Borough of Juneau. (2016).  \textit{Juneau’s Visitor Industry: Positive Economic Benefits for Our Community}. 
  \bibitem{jedc2024} Juneau Economic Development Council (JEDC). (2024).  \textit{Juneau SEAK Economic Indicators and Outlook Report}. 
  \bibitem{juneau2021} City and Borough of Juneau. (2021).  \textit{Juneau Tourism Survey 2021}. 
  \bibitem{juneau2022} City and Borough of Juneau. (2022).  \textit{Juneau Tourism Survey 2022}. 
  \bibitem{juneau2023} City and Borough of Juneau. (2023).  \textit{Juneau Tourism Survey 2023}. 
  \bibitem{circulator2024} City and Borough of Juneau. (2024).  \textit{Juneau Visitor Circulator Study: Final Report}. 
  \bibitem{budget2016} City and Borough of Juneau Assembly. (2016).  \textit{Biennial Budget Fiscal Year 2016: Year 2 of the FY15/FY16 Biennial Budget}. 
  \bibitem{budget2018} City and Borough of Juneau. (2018).  \textit{Biennial Budget Adopted Fiscal Year 2018: Year 2 of the FY17/18 Biennial Budget}. 
  \bibitem{budget2020} City and Borough of Juneau. (2020).  \textit{Biennial Budget Adopted Fiscal Year 2020: Year 2 of the FY19/20 Biennial Budget}. 
  \bibitem{budget2022} City and Borough of Juneau. (2022).  \textit{Biennial Budget Adopted Fiscal Year 2022: Year 2 of the FY21/22 Biennial Budget}. 
  \bibitem{budget2024} City and Borough of Juneau. (2024).  \textit{Biennial Budget Adopted Fiscal Year 2024: Year 2 of the FY23/24 Biennial Budget}. 
  \bibitem{alaskaInventory}Alaska Department of Environmental Conservation Division of Air Quality. (2023). \textit{Alaska Greenhouse Gas Emissions Inventory}.
  \bibitem{} Alaska Municipal League (2024). \textit{State of Alaska Priority Sustainable Energy Action Plan}
  \bibitem{} UK NIR (2008). \textit{Detailed discussion of methodology and data for estimating $\mathrm{CO}_2$ emissions from fossil fuel combustion}
  \bibitem{} Final Alaska Greenhouse Gas Inventory and Reference
  Case Projections (2009). \textit{Greenhouse Gas Emissions Inventory and Reference Case Projections}
\end{thebibliography}


\clearpage

\section*{\centering Report on Use of AI}

\subsection*{GPT4o}

Messages below are conversations between a human and GPT-4o. The human is represented by \textit{Q} and GPT-4o is represented by \textit{A}.

\vspace{0.5cm}

\textit{Q1}: Write me an introduction of Juneau, Alaska. Talk about its residents, tourists and attractions.

\textit{A1}: Juneau, Alaska, the state’s capital, is a city that blends breathtaking natural beauty with rich cultural heritage. Nestled in the southeastern part of the state, Juneau is uniquely accessible only by air or sea, giving it an island-like charm despite being part of the mainland. The city is home to approximately 30,000 residents who enjoy a lifestyle deeply connected to nature, with fishing, hiking, and outdoor exploration forming an integral part of daily life.

\vspace{0.5cm}

\textit{Q2}: Can you tell me about the tourism industry in Juneau?

\textit{A2}: Juneau is a popular destination for tourists from around the world, attracting over a million visitors annually. The city’s natural wonders, including glaciers, rainforests, and abundant wildlife, make it a magnet for outdoor enthusiasts and nature lovers. Tourists can explore the Mendenhall Glacier, go whale watching, or take a scenic flight over the stunning landscape. The city also offers a vibrant arts and culture scene, with museums, galleries, and theaters showcasing the region’s history and creativity.

\vspace{0.5cm}

\textit{Q3}: What does Thousands of Chained 2017 U.S. Dollars mean?

\textit{A3}: Thousands of Chained 2017 U.S. Dollars is a measure of economic value that adjusts for inflation and changes in purchasing power over time. By using the 2017 dollar as a base year, the measure allows for more accurate comparisons of economic data across different years, accounting for the effects of inflation on prices and wages.

\vspace{0.5cm}

\textit{Q4}: What is the Kaya Identity?

\textit{A4}: The Kaya Identity is a mathematical formula used to decompose the factors driving changes in carbon dioxide emissions. It breaks down emissions into four components: population, GDP per capita, energy intensity of the economy, and carbon intensity of energy. By analyzing these factors, policymakers can identify the key drivers of emissions growth and develop strategies to reduce carbon footprints.

\vspace{0.5cm}

\textit{Q5}: Cost of converting CO2 to non-greenhouse gases?

\textit{A5}: The cost of converting CO2 to non-greenhouse gases depends on the method used and the scale of the operation. Technologies such as carbon capture and storage (CCS) and direct air capture (DAC) can be expensive due to the energy and resources required to capture and convert CO2. However, as these technologies become more advanced and widespread, the cost of converting CO2 to non-greenhouse gases is expected to decrease, making it a more viable solution for reducing emissions.

\subsection*{GPT4o-mini}

Messages below are conversations between a human and GPT4o-mini. The human is represented by \textit{Q} and 
GPT4o-mini is represented by \textit{A}.
\vspace{0.5cm}

\textit{Q6}: What does MTCO2e mean?

\textit{A6}: MTCO2e stands for metric tons of carbon dioxide equivalent, a unit of measurement used to quantify the global warming potential of greenhouse gases. By expressing emissions in terms of CO2 equivalents, different gases can be compared based on their impact on the climate, allowing for a more comprehensive assessment of a country or region’s carbon footprint.

\vspace{0.5cm}

\textit{Q7}: How to quantify the damage humans have done to the environment?

\textit{A7}: Quantifying the damage humans have done to the environment is a complex and multifaceted task that requires a combination of scientific, economic, and social analyses. Environmental impact assessments, life cycle assessments, and ecological footprint analyses are some of the tools used to measure the effects of human activities on ecosystems, biodiversity, and natural resources. By evaluating factors such as pollution, habitat destruction, and resource depletion, researchers can estimate the extent of environmental damage and develop strategies to mitigate its impact.

\end{document}
