\begin{summary}
    Juneau, capital city of Alaska and home to approximately 30,000 residents, 
    is a unique city that welcomes over a million tourists annually. 
    The city boasts a number of natural treasures, including glaciers, 
    rain forests and whales. However, due to the flourishment of tourism, 
    Juneau now faces challenges such as receding glaciers, crowded streets,
    skyrocketing carbon emissions, and the loss of cultural heritage.
    This paper aims to propose a sustainable tourism strategy for Juneau and
    migrate the model to Sitka, Alaska. 

    Prior to the tasks, we analyzed our collected data and conducted a thorough 
    prediction of various factors that affect the tourism industry in Juneau.
    Models such as \textbf{ARIMA}, \textbf{Linear Regression} were used to predict
    the number of tourists, and \textbf{K-means} was used to cluster the data.

    In Task 1, we developed three models to respectively target at three categories: 
    the economy, environment, and society (hidden causes). For economy, we proposed 
    \textbf{Tourism Income Anslysis Model} to analyse the relation bwtween tax rate and 
    the number of tourists, \textbf{SARIMAX} was also used to predict data in the future. 
    For environment, we proposed \textbf{Kaya\_Tourism Model}, extending the original Kaya model,
    to analyse the carbon emissions especially from tourism. 
    For society, we proposed \textbf{Social\_Impact\_Model} and used \textbf{Entropy Weight Method (EWM)}
    to calculate the weight of each factor. 
    The models were then combined to form a comprehensive model to quantify the tourism industry in Juneau.
    We then found the optimal tax rate, number of tourists, and fine rate
    to maximize the tourism income and environmental sustainability. 

    We conducted a sensitivity analysis by \textbf{varying one factor at a time while keeping the others constant}, to evaluate the impact of each factor on the tourism industry.

    In Task 2, we adapted the model to Sitka, Alaska, and tested its adaptability and migration capability.
    We found that the model could be successfully adapted to Sitka, and the optimal tax rate, number of tourists, and fine rate were calculated.

    In Task 3, we proposed a sustainable tourism strategy for Juneau, in which we firstly introduced our
    model and then proposed a series of measures and recommendations to promote the 
    sustainable development of the tourism industry in Juneau.

    In conclusion, we have effectively built a model that can quantify the tourism industry in Juneau, adapted the model to another city, and proposed a sustainable tourism strategy for Juneau.

    \vspace{0.5cm}

    \textbf{Keywords:} Juneau, Sitka, SARIMAX, Kaya, Entropy Weight Method, PCA, sustainable tourism
\end{summary}
    