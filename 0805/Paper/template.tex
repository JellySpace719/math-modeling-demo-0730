%! Mode:: "TeX:UTF-8"
%! TEX program = xelatex
\PassOptionsToPackage{quiet}{xeCJK}
\documentclass[withoutpreface,bwprint]{cumcmthesis}
\usepackage{etoolbox}
\BeforeBeginEnvironment{tabular}{\zihao{-5}}
\usepackage[numbers,sort&compress]{natbib}  % 文献管理宏包
\usepackage[framemethod=TikZ]{mdframed}  % 框架宏包
\usepackage{url}  % 网页链接宏包
\usepackage{subcaption}  % 子图宏包
\newcolumntype{C}{>{\centering\arraybackslash}X}
\newcolumntype{R}{>{\raggedleft\arraybackslash}X}
\newcolumntype{L}{>{\raggedright\arraybackslash}X}

\usepackage{multirow}
\usepackage{array}

\title{全国大学生数学建模竞赛论文模板}  % 论文标题
\tihao{}  % 题号
\baominghao{}  % 报名号
\schoolname{}  % 学校
\membera{}  % 队员a
\memberb{}  % 队员b
\memberc{}  % 队员c
\supervisor{}  % 指导老师
\yearinput{}
\monthinput{}
\dayinput{}

%%%%%%%%%%%%%%%%%%%%%%%%%%%%%%%%%%%%%%%%%%%%%%%%%%%%%%%%%%%%%
%% 正文
\begin{document}
\begin{samepage}
\begin{center}
\renewcommand{\arraystretch}{2}
\begin{tabular}{|>{\centering\arraybackslash}m{0.13\textwidth}
                |>{\centering\arraybackslash}m{0.65\textwidth}
                |>{\centering\arraybackslash}m{0.17\textwidth}|}
    \hline
    \songti 所属类别
    & \multirow{2}{*}{\centering\songti\zihao{4} \textbf{2024年“华数杯”全国大学生数学建模竞赛}}
    & \songti 参赛编号 \\
    \cline{1-1} \cline{3-3}
    \songti 本科组 & & CM2400947 \\
    \hline
\end{tabular}
\end{center}

\maketitle

\begin{abstract}
    母亲的身心健康对婴儿的早期成长和发展具有重要影响。已有研究表明,母亲的身体状况以及心理健康(如压力、抑郁、焦虑等)不仅关系到婴儿的生理健康,还会影响婴儿的认知、情感、社会行为等多方面成长。婴儿的行为特征和睡眠质量常常作为衡量婴儿发展状况的重要指标。基于这方面的专业背景,本文对相关影响机制进行了研究与建模。
    
    \textbf{对于问题一,}本文通过分析390名婴儿及其母亲的身体和心理相关指标数据,研究了母亲因素对婴儿行为特征和睡眠质量的影响关系。
    
    \textbf{对于问题二,}基于婴儿行为问卷,将婴儿行为特征划分为安静型、中等型和矛盾型,建立了行为特征与母亲身体、心理指标之间的数学关系模型,并利用该模型预测了20组缺失婴儿行为类型。
    
    \textbf{对于问题三,}建立了CBTS、EPDS、HADS评分与治疗费用的关联模型,针对编号238的矛盾型婴儿,分析通过最小治疗费用使其行为特征由矛盾型转变为中等型和安静型的策略及调整路径。
    
    \textbf{对于问题四,}选取婴儿睡眠的多项指标,综合评判睡眠质量,并建立母亲各项指标与婴儿综合睡眠质量的关联模型,进而预测缺失数据婴儿的睡眠类别。
    
    最后,结合模型结果,对如何提升238号婴儿的睡眠质量等级为优提出了具体的治疗方案和优化建议。
    
    \keywords{母亲心理健康\quad  婴儿成长\quad  行为特征\quad  睡眠质量 \quad 数学建模}
    \end{abstract}
\end{samepage}
%%%%%%%%%%%%%%%%%%%%%%%%%%%%%%%%%%%%%%%%%%%%%%%%%%%%%%%%%%%%% 

% \tableofcontents  % 目录
% \newpage

%%%%%%%%%%%%%%%%%%%%%%%%%%%%%%%%%%%%%%%%%%%%%%%%%%%%%%%%%%%%%  
\section{问题重述}
\subsection{问题背景}
问题背景

%%%%%%%%%%%%%%%%%%%%%%%%%%%%%%%%%%%%%%%%%%%%%%%%%%%%%%%%%%%%% 

\subsection{问题要求}

\textbf{问题1}  

\textbf{问题2}  

\textbf{问题3} 

\textbf{问题4}  

%%%%%%%%%%%%%%%%%%%%%%%%%%%%%%%%%%%%%%%%%%%%%%%%%%%%%%%%%%%%% 

\section{问题分析}
\subsection{问题一分析}
对于问题一,

\subsection{问题二分析}	
对于问题二,

\subsection{问题三分析}
对于问题三,

\subsection{问题四分析}
对于问题四,

%%%%%%%%%%%%%%%%%%%%%%%%%%%%%%%%%%%%%%%%%%%%%%%%%%%%%%%%%%%%% 

\section{模型假设}

为简化问题,本文做出以下假设:

\begin{itemize}[itemindent=2em]
\item 假设1
\item 假设2
\item 假设3
\end{itemize}

%%%%%%%%%%%%%%%%%%%%%%%%%%%%%%%%%%%%%%%%%%%%%%%%%%%%%%%%%%%%% 

\section{符号说明}
\begin{table}[H]
\centering
\begin{tabularx}{\textwidth}{CLC}
\toprule
符号    & 说明    & 单位 \\
\midrule
$m     $& 质量 & $kg$ \\
$V     $& 体积 & $m^3$ \\
\bottomrule
\end{tabularx}
\label{tab:符号说明}
\end{table}


%%%%%%%%%%%%%%%%%%%%%%%%%%%%%%%%%%%%%%%%%%%%%%%%%%%%%%%%%%%%% 

\section{问题一的模型的建立和求解}
\subsection{模型建立}

$$
E = mc^2
$$

引用公式\cref{eq:公式1}。

\begin{equation}
\label{eq:公式1}
E = mc^2
\end{equation}

引用\cref{fig:单图}。

\begin{figure}[ht]
\centering
\includegraphics[width=0.75\textwidth]{example.eps}
\caption{单图}
\label{fig:单图}
\end{figure}

这句话引用了文献\cite{司守奎2011数学建模算法与应用}。

这句话引用了文献\upcite{卓金武2011MATLAB}。

\subsection{模型求解}

\textbf{Step1:} 

\textbf{Step2:} 

\textbf{Step3:} 

\subsection{求解结果}


%%%%%%%%%%%%%%%%%%%%%%%%%%%%%%%%%%%%%%%%%%%%%%%%%%%%%%%%%%%%% 

\section{问题二的模型的建立和求解}
\subsection{模型建立}

引用\cref{fig:双图},引用\cref{fig:双图a},引用\cref{fig:双图b}。

\begin{figure}[ht]
\centering
\subcaptionbox{双图a子标题\label{fig:双图a}}
{\includegraphics[width=.4\textwidth]{example.eps}}
\subcaptionbox{双图b子标题\label{fig:双图b}}
{\includegraphics[width=.4\textwidth]{example.eps}}
\caption{双图}\label{fig:双图}
\end{figure} 

\subsection{模型求解}

\textbf{Step1:} 

\textbf{Step2:} 

\textbf{Step3:} 

\subsection{求解结果}

%%%%%%%%%%%%%%%%%%%%%%%%%%%%%%%%%%%%%%%%%%%%%%%%%%%%%%%%%%%%% 

\section{问题三的模型的建立和求解}
\subsection{模型建立}

\subsection{模型求解}

\textbf{Step1:} 

\textbf{Step2:} 

\textbf{Step3:} 

\subsection{求解结果}

%%%%%%%%%%%%%%%%%%%%%%%%%%%%%%%%%%%%%%%%%%%%%%%%%%%%%%%%%%%%% 

\section{问题四的模型的建立和求解}
\subsection{模型建立}

\subsection{模型求解}

\textbf{Step1:} 

\textbf{Step2:} 

\textbf{Step3:} 

\subsection{求解结果}

%%%%%%%%%%%%%%%%%%%%%%%%%%%%%%%%%%%%%%%%%%%%%%%%%%%%%%%%%%%%%

\section{模型的分析与检验}

\subsection{灵敏度分析}

\subsection{误差分析}

%%%%%%%%%%%%%%%%%%%%%%%%%%%%%%%%%%%%%%%%%%%%%%%%%%%%%%%%%%%%%

\section{模型的评价}

\subsection{模型的优点}
\begin{itemize}[itemindent=2em]
\item 优点1
\item 优点2
\item 优点3
\end{itemize}

\subsection{模型的缺点}
\begin{itemize}[itemindent=2em]
\item 缺点1
\item 缺点2
\end{itemize}

%%%%%%%%%%%%%%%%%%%%%%%%%%%%%%%%%%%%%%%%%%%%%%%%%%%%%%%%%%%%%
%% 参考文献
\nocite{*}
\bibliographystyle{gbt7714-numerical}  % 引用格式
\bibliography{ref.bib}  % bib源

\newpage
%%%%%%%%%%%%%%%%%%%%%%%%%%%%%%%%%%%%%%%%%%%%%%%%%%%%%%%%%%%%%
%% 附录
\begin{appendices}
\section{文件列表}
\begin{table}[H]
\centering
\begin{tabularx}{\textwidth}{LL}
\toprule
文件名   & 功能描述 \\
\midrule
q1.m & 问题一程序代码 \\
q2.py & 问题二程序代码 \\
q3.c & 问题三程序代码 \\
q4.cpp & 问题四程序代码 \\
\bottomrule
\end{tabularx}
\label{tab:文件列表}
\end{table}

\section{代码}
\noindent q1.m
\lstinputlisting[language=matlab]{code/q1.m}
q2.py
\lstinputlisting[language=python]{code/q2.py}
q3.c
\lstinputlisting[language=c]{code/q3.c}
q4.cpp
\lstinputlisting[language=c++]{code/q4.cpp}
\end{appendices}
\end{document}


%%%%%双图模板%%%%%%
\begin{figure}
\centering
\subcaptionbox{炉温曲线示意图\label{fig:双图a}}
{\includegraphics[width=.4\textwidth]{炉温曲线示意图.png}}
\subcaptionbox{问题1炉温曲线\label{fig:双图b}}
{\includegraphics[width=.4\textwidth]{问题1炉温曲线.png}}
\caption{双图}\label{fig:双图}
\end{figure} 
%%%%%双图模板%%%%%%