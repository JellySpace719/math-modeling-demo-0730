\section*{\centering Report on Use of AI}

\subsection*{GPT4o}

Messages below are conversations between a human and GPT-4o. The human is represented by \textit{Q} and GPT-4o is represented by \textit{A}.

\vspace{0.5cm}

\textit{Q1}: Write me an introduction of Juneau, Alaska. Talk about its residents, tourists and attractions.

\textit{A1}: Juneau, Alaska, the state’s capital, is a city that blends breathtaking natural beauty with rich cultural heritage. Nestled in the southeastern part of the state, Juneau is uniquely accessible only by air or sea, giving it an island-like charm despite being part of the mainland. The city is home to approximately 30,000 residents who enjoy a lifestyle deeply connected to nature, with fishing, hiking, and outdoor exploration forming an integral part of daily life.

\vspace{0.5cm}

\textit{Q2}: Can you tell me about the tourism industry in Juneau?

\textit{A2}: Juneau is a popular destination for tourists from around the world, attracting over a million visitors annually. The city’s natural wonders, including glaciers, rainforests, and abundant wildlife, make it a magnet for outdoor enthusiasts and nature lovers. Tourists can explore the Mendenhall Glacier, go whale watching, or take a scenic flight over the stunning landscape. The city also offers a vibrant arts and culture scene, with museums, galleries, and theaters showcasing the region’s history and creativity.

\vspace{0.5cm}

\textit{Q3}: What does Thousands of Chained 2017 U.S. Dollars mean?

\textit{A3}: Thousands of Chained 2017 U.S. Dollars is a measure of economic value that adjusts for inflation and changes in purchasing power over time. By using the 2017 dollar as a base year, the measure allows for more accurate comparisons of economic data across different years, accounting for the effects of inflation on prices and wages.

\vspace{0.5cm}

\textit{Q4}: What is the Kaya Identity?

\textit{A4}: The Kaya Identity is a mathematical formula used to decompose the factors driving changes in carbon dioxide emissions. It breaks down emissions into four components: population, GDP per capita, energy intensity of the economy, and carbon intensity of energy. By analyzing these factors, policymakers can identify the key drivers of emissions growth and develop strategies to reduce carbon footprints.

\vspace{0.5cm}

\textit{Q5}: Cost of converting CO2 to non-greenhouse gases?

\textit{A5}: The cost of converting CO2 to non-greenhouse gases depends on the method used and the scale of the operation. Technologies such as carbon capture and storage (CCS) and direct air capture (DAC) can be expensive due to the energy and resources required to capture and convert CO2. However, as these technologies become more advanced and widespread, the cost of converting CO2 to non-greenhouse gases is expected to decrease, making it a more viable solution for reducing emissions.

\subsection*{GPT4o-mini}

Messages below are conversations between a human and GPT4o-mini. The human is represented by \textit{Q} and 
GPT4o-mini is represented by \textit{A}.
\vspace{0.5cm}

\textit{Q6}: What does MTCO2e mean?

\textit{A6}: MTCO2e stands for metric tons of carbon dioxide equivalent, a unit of measurement used to quantify the global warming potential of greenhouse gases. By expressing emissions in terms of CO2 equivalents, different gases can be compared based on their impact on the climate, allowing for a more comprehensive assessment of a country or region’s carbon footprint.

\vspace{0.5cm}

\textit{Q7}: How to quantify the damage humans have done to the environment?

\textit{A7}: Quantifying the damage humans have done to the environment is a complex and multifaceted task that requires a combination of scientific, economic, and social analyses. Environmental impact assessments, life cycle assessments, and ecological footprint analyses are some of the tools used to measure the effects of human activities on ecosystems, biodiversity, and natural resources. By evaluating factors such as pollution, habitat destruction, and resource depletion, researchers can estimate the extent of environmental damage and develop strategies to mitigate its impact.