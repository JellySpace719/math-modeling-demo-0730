\section{问题1:两组专家评分的显著性差异及可信度分析}

\subsection{问题背景与研究意义}

\indent 在高校教师职称评定过程中,课堂教学评分是衡量教师教学质量的重要依据。传统上,学校通常由教务处或教师发展中心聘请有资质的专家团队,对候选教师每门课程的教学表现进行多维度评分并加总总分,作为职称或评优决策的关键参考。然而,评分专家的组成、评分尺度及习惯差异,均可能导致结果的主观偏差。随着参评教师数量逐年增加,为提高评价效率和应对管理压力,部分高校尝试分层委托学院自主组织评价。这在降低中心工作负担的同时,也带来了评分标准不一致等新挑战。

为了提升评价体系的科学性和公信力,需要系统考察专家评分的一致性和可靠性。附件1中,2023年某校采用统一组织方式安排了两组独立专家队伍对同一批教师进行课堂评价。基于此数据,本文聚焦以下两点问题:
\begin{enumerate}
    \item 两组专家对同批教师的评分结果之间是否存在统计学上的显著性差异?
    \item 哪一组专家评分结果在信度和一致性上更值得信赖?
\end{enumerate}

\subsection{模型假设}
为保障分析的科学性,本文做出如下建模假设:
\begin{enumerate}
    \item 每位教师都获得了两组专家的评分数据,数据配对且完整无缺失。
    \item 专家评分已充分反映各项教学细则,均以加权总分形式呈现。
    \item 两组专家互为独立评价主体,评分时各自依据学校统一指标体系操作,互不影响。
    \item 两组评分数据及差值近似服从正态分布,若正态性假设不满足,则补充采用非参数检验方法。
\end{enumerate}

\subsection{变量与符号说明}
\begin{itemize}
    \item $n$:参评教师的人数。
    \item $S^{(1)}_i$:第$i$位教师第一组专家的平均总分。
    \item $S^{(2)}_i$:第$i$位教师第二组专家的平均总分。
    \item $D_i = S^{(1)}_i - S^{(2)}_i$:第$i$位教师在两组专家评分中的差值。
\end{itemize}

\subsection{分析流程与统计建模}
\subsubsection{数据预处理与描述性分析}
首先,梳理并计算每位教师的两组专家平均总分,并对各组数据进行均值、标准差、最小值、最大值等基础统计量的提取。通过绘制箱线图、直方图等可视化手段,直观比较两组分布的形态特征以及离群点信息,为后续分析提供支持。

\subsubsection{配对显著性检验}
因两组估分针对同一批对象,构成配对样本。为检验谁是否存在系统性均值差异,需首先对差值$D_i$进行正态性检验(如Shapiro-Wilk)。若满足正态假设,采用配对$t$检验方法;否则,则改用Wilcoxon符号秩检验。检验结论以$p$值为准,$p<0.05$视为两组评分在统计学上显著不同;$p\geq 0.05$则差异不显著。

\subsubsection{效应量评估}
除显著性检验外,还需判定平均得分差异的实际影响。采用Cohen's $d$作为效应量度量,公式如下:
\[
d = \frac{\overline{D}}{s_D}
\]
其中$\overline{D}$为所有差值的均值,$s_D$为其标准差。$|d|\approx0.2$表示小效应,$\approx0.5$为中效应,$\approx0.8$以上为大效应,便于对统计差异的实际意义进行判定。

\subsubsection{专家组内一致性(信度)分析}
可信度评估采用组内相关系数(Intraclass Correlation Coefficient,ICC),定量衡量各专家在同组内对教师打分的一致性。ICC值越高,说明评分共识性越强,评价越稳定可靠。ICC常用判断标准为:低于0.5为一致性差,0.5-0.75为中等,0.75-0.9为良好,高于0.9为优秀。

\subsubsection{多维判据综合}
通过差异性显著性检验、效应量分析和ICC信度比较,综合判定:
\begin{itemize}
    \item 两组专家评分结果是否有统计学和实际意义上的显著差异。
    \item 哪组评分更具内部一致性和可信度,适合作为最终评价参考。
\end{itemize}

\subsection{结果与讨论(模板示例,后续需替换真实数据)}
例如,假如分析得出第一组评分均值高于第二组,配对$t$检验$p=0.03$,Cohen's $d=0.35$(小效应),第一组ICC为0.89,第二组ICC为0.76,则可得:
\begin{quote}
统计分析表明,两组专家评分均值在统计上具有显著差异($p=0.03<0.05$),但实际效应量较低(Cohen's $d=0.35$)。更进一步,第一组专家评分的一致性更强(ICC=0.89>0.76),显示评分标准更为集中,同组专家间共识度更高。因此,建议采纳第一组的评分结果作为后续评价的重要依据。
\end{quote}

\subsection{小结}
本文通过配对样本检验、效应量度量和一致性信度分析,系统评估了两组专家评分的差异及可信性。方法不仅揭示了评分差异的统计意义和实际影响,还为后续教师评价数据的标准化与权威性奠定了定量基础。该流程对科学、公正地优化高校教学评价体系具有重要参考意义。