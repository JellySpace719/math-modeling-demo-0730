\section{问题一:两组专家对同批教师教学评价的差异性与可信度分析}

\subsection{问题重述}
教师课堂教学评价是高校职称评定体系的重要组成部分。
在高校教师职称评定过程中,课堂教学评分是衡量教师教学质量的重要依据。传统上,学校通常由教务处或教师发展中心聘请有资质的专家团队,对候选教师逐门课程的课堂表现进行多维度评分并加总总分,作为职称与评优决策参考。然而,评分专家组成、评价尺度及个人打分习惯的不同,可能导致一定程度的主观偏差。随着参评教师基数逐年增加,部分高校为提升评价效率、缓解管理压力,尝试将部分评价任务下放到各学院自评。此举在减轻负担的同时,也产生了评分标准不统一等新问题。
2023年,学校集中组织了两组专家分别对同一批教师进行了多指标评分。本文需分析:
\begin{enumerate}
    \item 两组专家对同批教师的评分结果之间是否存在统计学上的显著性差异?
    \item 哪一组专家评分结果在信度和一致性上更值得信赖?
\end{enumerate}
\subsection{建模思路与分析流程}
本节思路如下:

\begin{enumerate}
    \item \textbf{数据预处理与批量求和}:针对原始多层次评分表,对每位教师、每组专家的11项具体指标分数求和,得出教师—专家—组的总分数据。对于极个别缺失情况,采用“同一教师该指标其他专家均值补齐”策略,确保数据完整性。
    \item \textbf{描述性统计分析}:对两组专家评分总分计算均值、中位数、标准差、极差、偏度、峰度等,并通过箱线图、直方图展示分布,为后续检验提供支持。
    \item \textbf{显著性差异检验与效应量分析}:考虑到数据为同一批教师的配对评分,首先进行差值的正态性检验(Shapiro-Wilk),如通过则用配对t检验,否则用Wilcoxon符号秩检验。进而计算Cohen's $d$,定量衡量实际意义。
    \item \textbf{内部一致性与可信度评价}:采用组内相关系数ICC(双向随机效应、绝对一致性、单次测量模型 ICC(2,1)),分析每组专家评分的一致性,值越大可信度越高。
    \item \textbf{综合判据与结论}:整合差异显著性、效应大小与ICC,系统比较两组专家数据的可靠性。
\end{enumerate}

\subsection{数据处理与描述性统计}


\begin{itemize}
    \item $n$:参评教师人数。
    \item $S^{(1)}_i$:第$i$位教师第一组专家的平均总分。
    \item $S^{(2)}_i$:第$i$位教师第二组专家的平均总分。
    \item $D_i = S^{(1)}_i - S^{(2)}_i$:第$i$位教师两组评分平均值差异。
\end{itemize}

\begin{figure}[htbp]
    \centering
    \includegraphics[width=0.6\textwidth]{scores_scatter.png} % 插入图片
    \caption{散点图}
\end{figure}


统计并对比两组评分的均值、标准差、极差、偏度、峰度等核心指标,并辅以箱线图、直方图等可视化手段直观展示分布形态、中心趋势、离散程度和潜在异常,为后续推断检验和一致性评估奠定基础。

\begin{figure}[htbp]
    \centering
    \includegraphics[width=1\textwidth]{descriptive_statistics.png} % 插入图片
    \caption{直方图和箱线图}
\end{figure}



对所有教师-专家数据,经清洗和缺失项均值填补后,分别求得两组专家每位教师的平均总分 $S^{(1)}_i, S^{(2)}_i$。以附件1为例,第一组专家评分均值88.54,标准差4.34,第二组均值86.18,标准差5.35,均略呈负偏态。表~\ref{tab:descstat} 总结描述性统计结果。



\begin{table}[h]
\centering
\caption{两组专家评分描述性统计}
\label{tab:descstat}
\begin{tabular}{lcc}
\toprule
统计指标 & 第一组专家评分 & 第二组专家评分 \\
\midrule
样本量      & 50        & 50        \\
均值        & 88.54     & 86.18     \\
标准差      & 4.34      & 5.35      \\
最小值      & 77.00     & 76.00     \\
极差        & 20.90     & 23.00     \\
偏度        & -0.23     & 0.23      \\
峰度        & -0.05     & -0.68     \\
\bottomrule
\end{tabular}
\end{table}

\subsection{显著性差异检验与效应量}
\subsubsection{配对显著性检验}
针对同一批教师的配对评分,先对两组评分差值$D_i$进行Shapiro-Wilk正态性检验。若近似正态,采用配对$t$检验对两组均值差异做统计推断;若非正态,则采用Wilcoxon符号秩检验。以$p$值小于0.05为统计显著,结合效应量判断实际意义。

\subsubsection{效应量量化}
采用配对样本下的Cohen's $d$量化两组专家均值差异的实际影响:
\[
d = \frac{\overline{D}}{s_D}
\]
其中$\overline{D}$为所有差值均值,$s_D$为样本标准差。$|d| \approx 0.2$为小效应,$\approx 0.5$为中等效应,$\approx 0.8$为大效应,这可弥补$p$值无法揭示实际影响大小的不足。

教师的两组专家平均分之差 $D_i = S^{(1)}_i - S^{(2)}_i$ 经Shapiro-Wilk检验$p=0.187$,近似正态,适合配对样本t检验。t检验$p=0.001$,两组专家评分存在统计学显著差异。Cohen's $d=0.52$,为中等效应,说明分组差异在实际教学评价中具备一定意义。

\subsection{专家组可靠性(内部一致性)分析}
采用组内相关系数ICC(Intraclass Correlation Coefficient)定量评价组内一致性。模型选用“双向随机效应、绝对一致性、单次测量”ICC(2,1)标准,是因为:

\begin{itemize}
    \item 双向随机效应假设专家和被评分教师均为总体的随机抽取,增强结论的普适性;
    \item 绝对一致性着重考察分数本身(非仅排名)的一致水平;
    \item 单次测量聚焦个别评分者一致性(适合专家组实际结构)。
\end{itemize}

ICC判据:$<0.5$为差;$0.5{\sim}0.75$为中等;$0.75{\sim}0.9$良好;$\geq0.9$优秀。

\begin{figure}[H]
    \centering
    \begin{minipage}[t]{0.5\textwidth}
        \centering
        \includegraphics[width=1\textwidth]{icc_analysis.png}
    \end{minipage}
    \hfill
    \begin{minipage}[t]{1\textwidth}
        \centering
        \includegraphics[width=1\textwidth]{normality_test.png}
    \end{minipage}
\end{figure}


分别计算两组专家内ICC(双向随机效应、绝对一致性、单次测量):
\begin{itemize}
    \item 第一组ICC=0.518(中等一致性)
    \item 第二组ICC=0.323(差/不可接受的一致性)
\end{itemize}
第一组专家在评分标准和评分稳定性方面表现更优,具有更高的一致性和集体信度。

\subsection{综合判断与结论}
以实际分析为例,经描述性统计,第一组均值高于第二组(88.54 vs 86.18),标准差更低。差值$D_i$正态性检验$p=0.187$,支持采用配对$t$检验,检验$p=0.001$,存在显著差异。Cohen's $d=0.522$,为中等效应。

ICC分析显示,第一组ICC=0.518(中等一致性),第二组ICC=0.323(偏低一致性)。表明第一组专家判分标准更统一、一致性更优,具备更高集体信度。

本节通过多层次分析证明,两组专家对同批教师的打分不仅分布中心存在统计显著差异($p=0.001$),而且差异效应为中等水平。同时,内部一致性检验结果显示,第一组专家的评分标准更统一,输出更稳定,因此结果更具可信度。建议在教师评价体系优化和赋分权重调整时,优先参考一致性更高的专家组评分。





\subsection{小结}
本文针对两组专家对同批教师的评分用配对t检验、效应量和ICC信度分析,科学评估了评分差异和组内一致性。研究流程严谨,既揭示了分组间差异的统计和实际意义,也为后续高校教学评价权威化、标准化提供了数据与思路支撑。





