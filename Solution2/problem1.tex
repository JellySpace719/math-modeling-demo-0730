\documentclass[12pt]{article}
\usepackage[UTF8]{ctex}  % 中文支持
\usepackage{geometry}
\geometry{left=2.5cm,right=2.5cm,top=2.5cm,bottom=2.5cm}
\usepackage{amsmath,amssymb,amsthm}
\usepackage{booktabs}
\usepackage{array}
\usepackage{float}
\usepackage{enumitem}
\usepackage{xcolor}

% 设置标题格式
\usepackage{titlesec}
\titleformat{\section}{\Large\bfseries}{\thesection}{1em}{}
\titleformat{\subsection}{\large\bfseries}{\thesubsection}{1em}{}
\titleformat{\subsubsection}{\normalsize\bfseries}{\thesubsubsection}{1em}{}

% 设置行距
\usepackage{setspace}
\onehalfspacing

\title{问题一:景点评分最高分及分布的量化分析}
\author{}
\date{}

\begin{document}

\maketitle

\section{问题域界定与目标函数}

本研究的起始阶段,即问题一,旨在对中国境内(不含港澳台)352个城市所涵盖的\textbf{35200个离散信息实体(即旅游景点)}的内在评价机制进行\textbf{宏观量化分析}。核心目标在于:

\begin{enumerate}
    \item \textbf{全局极值识别:}确定所有景点评价指标(景点评分)集合中的\textbf{全局上确界(Supremum)},我们称之为"最优基准评分"(BestScore,简称$BS$)。
    \item \textbf{基准达成度量:}量化在整个数据集拓扑中,有多少个信息实体达到了这一全局最优基准。
    \item \textbf{空间分区优势评估:}识别并排序拥有最多达到最优基准评分信息实体的\textbf{地理空间分区(即城市)},并输出前10个具有显著优势的城市。
\end{enumerate}

此阶段的任务,虽表面为基础统计,实则为后续复杂决策模型(如多目标路径优化)奠定数据基础和提供关键参数。它通过对原始数据流的结构化解析与聚合,揭示了目标变量(景点评分)的内在分布特性及其在不同空间分区中的异质性表现。

\section{模型建立与算法范式}

鉴于本问题聚焦于数据集的内在统计属性挖掘,其"模型"并非传统意义上的预测或优化模型,而是一种\textbf{数据驱动的、基于迭代聚合的描述性分析范式}。我们旨在通过形式化定义关键变量和操作流程,将直观的统计需求转化为可计算的算法逻辑。

\subsection{形式化定义与数学符号}

设所有景点的集合为 $\mathcal{A} = \{a_k\}_{k=1}^{35200}$,其中 $a_k$ 表示第 $k$ 个景点。
每个景点 $a_k$ 关联一个评价指标 $S(a_k) \in \mathbb{R}$,即其景点评分。

\begin{enumerate}
    \item \textbf{最优基准评分($BS$):}
    \begin{equation}
    BS = \sup \{S(a_k) \mid a_k \in \mathcal{A}\}
    \end{equation}
    此定义捕捉了所有景点评分的最大值,作为评价质量的全局参考点。

    \item \textbf{全局基准达成实体计数($N_{BS\_Total}$):}
    \begin{equation}
    N_{BS\_Total} = |\left\{a_k \in \mathcal{A} \mid S(a_k) = BS\right\}|
    \end{equation}
    此为达到全局最优基准的景点集合的势(Cardinality)。

    \item \textbf{城市空间分区与局部基准达成计数($Count_{c}(BS)$):}
    
    设城市集合为 $\mathcal{C} = \{c_j\}_{j=1}^{352}$。
    对于任意城市 $c_j \in \mathcal{C}$,与其关联的景点子集为 $\mathcal{A}_{c_j} \subset \mathcal{A}$,且 $\bigcup_{j=1}^{352} \mathcal{A}_{c_j} = \mathcal{A}$。
    则对于每个城市 $c_j$,其局部基准达成计数定义为:
    \begin{equation}
    Count_{c_j}(BS) = |\left\{a_k \in \mathcal{A}_{c_j} \mid S(a_k) = BS\right\}|
    \end{equation}
    此度量反映了特定地理分区内高品质旅游资源的富集程度。
\end{enumerate}

最终目标是通过对 $Count_{c_j}(BS)$ 进行降序排列,揭示具有最高数量 $BS$ 景点的前10个城市。

\subsection{求解策略与计算架构}

本问题的求解采用\textbf{两阶段迭代聚合策略},并基于Python这一高级编程语言构建\textbf{计算框架}。核心库 \texttt{pandas} 被选定为处理大规模表格化数据的首选工具,其提供的数据结构(DataFrame)和矢量化操作极大地提升了数据处理的效率和鲁棒性。\texttt{os} 和 \texttt{glob} 库则负责底层文件系统交互与数据源的动态发现。

\textbf{计算流程细化:}

\begin{enumerate}
    \item \textbf{数据源摄入与预处理层(Data Ingestion and Preprocessing Layer):}
    \begin{itemize}
        \item \textbf{文件系统遍历:}利用 \texttt{glob.glob()} 函数执行对指定数据目录(\texttt{data\_dir})内所有 \texttt{.csv} 格式文件的\textbf{递归扫描},构建异构数据源的路径列表。
        \item \textbf{并行化数据流处理(概念):}尽管此处为串行处理,但其逻辑可扩展至并行。每个CSV文件代表一个城市的数据分区。
        \item \textbf{数据结构实例化:}初始化 \texttt{all\_scores} 列表作为承载所有景点评分的\textbf{一维特征向量},以及 \texttt{city\_bs\_count} 字典作为\textbf{键值对形式的聚合存储器}。
        \item \textbf{鲁棒性数据类型转换:}对于每个读取的 \texttt{DataFrame},对"景点评分"列执行 \texttt{pd.to\_numeric()} 强制类型转换。关键在于使用 \texttt{errors='coerce'} 参数,其作用是将无法解析为数值的异常字符自动\textbf{胁迫(coerce)}为 \texttt{NaN}(Not a Number)浮点值。这是一种\textbf{隐式的异常值处理机制},确保了数值型数据的纯净性,从而避免了在后续算术操作中因数据类型不一致而引起的程序中断。
        \item \textbf{有效数据提取:}仅将 \texttt{NaN} 值排除后的有效评分数据追加到 \texttt{all\_scores} 特征向量中。
    \end{itemize}

    \item \textbf{第一阶段:全局极值识别(Global Extremum Identification Phase):}
    
    在完成所有城市数据文件的初始遍历和评分数据聚合后,对\textbf{全局评分特征向量} \texttt{all\_scores} 应用数学上的 \texttt{max()} 运算。这一操作在计算上是 $O(N)$ 复杂度的,其中 $N$ 为景点总数,能够高效地定位到 $BS$。

    \item \textbf{第二阶段:分层基准达成度量与空间分区排序(Stratified Benchmark Achievement Measurement and Spatial Partition Ranking):}
    \begin{itemize}
        \item 以第一阶段确定的 $BS$ 为基准,执行\textbf{第二次数据源遍历}。
        \item \textbf{局部计数聚合:}对每个城市的数据分区,应用\textbf{布尔掩码(Boolean Masking)}筛选出评分精确等于 $BS$ 的景点,并计算其\textbf{集合的势}(即数量)。此操作同样在 \texttt{pandas} 的矢量化引擎中高效完成。
        \item \textbf{全局累加:}将每个城市的局部计数累加,得到 $N_{BS\_Total}$。
        \item \textbf{排序与截断:}将 \texttt{city\_bs\_count} 字典转化为键值对列表,并根据每个城市所拥有的 $BS$ 景点数量进行\textbf{降序排序}。最终,通过\textbf{切片操作}(\texttt{[:10]})提取并呈现排序结果中的前10个城市,完成对空间分区优势的量化排名。
    \end{itemize}
\end{enumerate}

这种两阶段的计算范式,确保了在确定全局最优基准后,能够精确地在每个局部空间分区上进行符合该基准的度量,最终实现对整个数据集的系统性解析。

\section{运算结果与数据洞察}

通过上述严谨的计算架构对提供的数据集进行处理,我们获得了以下关键的量化结果:

\begin{enumerate}
    \item \textbf{最高景点评分(BestScore, BS):}
    
    经验证,所有35200个景点评分中的全局上确界 $BS$ 值为:$5.0$。这表明在给定的评分体系中,满分是可达到的,并作为最高质量的参照点。

    \item \textbf{全国达成最优基准的景点总数($N_{BS\_Total}$):}
    
    经统计,全国共有 $2563$ 个景点成功达到了 $BS=5.0$ 的最优基准。这一数字揭示了中国旅游资源中,有显著一部分在游客评价体系中获得了最高认可。

    \item \textbf{高基准达成度空间分区(城市)的分布:}
    
    在对各城市所拥有的 $BS$ 景点数量进行排序后,我们得出了以下具有最高基准达成度景点数量的前10个城市:
\end{enumerate}

\begin{table}[H]
\centering
\caption{拥有最高评分(BS)景点数量最多的前10个城市}
\label{tab:top10cities}
\begin{tabular}{ccc}
\toprule
排名 & 城市名称 & 拥有BS景点数量(个) \\
\midrule
1    & 三沙     & 36                   \\
2    & 五家渠   & 28                   \\
3    & 玉溪     & 21                   \\
4    & 益阳     & 20                   \\
5    & 天门     & 19                   \\
6    & 大兴安岭 & 18                   \\
7    & 潍坊     & 18                   \\
8    & 烟台     & 18                   \\
9    & 阿拉尔   & 18                   \\
10   & 邢台     & 17                   \\
\bottomrule
\end{tabular}
\end{table}

\textbf{数据洞察:}

分析此结果,我们发现\textbf{三沙市}以其卓越的 36 个 $BS$ 景点,显著超越其他城市,占据了此项指标的榜首。这可能反映了其独特的自然生态环境和旅游体验,在游客群体中形成了高度集中的正面评价。值得注意的是,前10榜单中不乏一些非传统意义上的热门旅游大城市,如五家渠、益阳、天门等。这一现象暗示着,在"景点评分"这一单一维度上,\textbf{区域性特色景点、独特体验或小众目的地,可能在游客满意度上具备与一线城市核心景区相媲美甚至超越的潜力}。此初步的定量分析为后续问题中对城市进行多维度、综合性的评价提供了重要的基础数据支撑和初级洞察,强调了景点本身质量在吸引游客方面不可或缺的作用。

\end{document}